\documentclass[12pt]{article}

\usepackage[utf8]{inputenc}

% TikZ tiene un conflicto con Babel, específicamente con el idioma español.
% Este conflicto se puede resolver cambiando esta línea de código.
\usepackage[spanish,es-noshorthands]{babel}

\usepackage[legalpaper, margin=2cm]{geometry}
\usepackage{amsmath, amsfonts, amssymb}
\usepackage{graphicx, colortbl, fancybox}

% Para poder hacer dibujos en TikZ, es necesario importar el paquete
\usepackage{tikz}

\setlength{\parskip}{1em}

\title{Mi Documento}
\author{Autor}
\date{\today}

\begin{document}

  \section*{TikZ}

    TikZ es un lenguaje en \LaTeX{} que nos permitirá crear figuras, formas,
    diagramas, grafos, etc. Esto aumenta las capacidades de \LaTeX{} para
    representar figuras.

    \begin{center}

      % Para comenzar un entorno de TikZ, se debe usar esta sección
      \begin{tikzpicture}
        \filldraw[color=red!80, fill=red!25, very thick](0,0) rectangle (3,2);
        \draw[orange, ultra thick] (4,0) -- (6,0) -- (5.7,2) -- cycle;
        \fill[blue!50] (8,1) circle (1.5);
      \end{tikzpicture}

    \end{center}

    Los comandos TikZ se pueden usar de forma separada:

    \tikz \draw (0,0) circle (1);
    \tikz \draw (-1,0) -- (0,1) -- (1,0) -- (0,-1) -- cycle;

    O se pueden usar en un mismo ``lienzo'':

    \begin{tikzpicture}
      \draw (0,0) circle (1cm);
      \draw (-1,0) -- (0,1) -- (1,0) -- (0,-1) -- cycle;
    \end{tikzpicture}

\end{document}
