\documentclass[12pt]{article}

\usepackage[utf8]{inputenc}
\usepackage[spanish]{babel}
\usepackage[letterpaper]{geometry}

% Para escribir simbolos matematicos, se necesitan estos paquetes
\usepackage{amsmath, amsfonts, amssymb}

\setlength{\parskip}{1em}

\begin{document}

    % \( ... \) se usa para poner expresiones matematicas "en linea" con el texto
    \LaTeX tambien se caracteriza por poder poner expresiones matematicas
    pero de una forma relativamente sencilla.

    % \ ... \] se usa para poner expresiones matematicas "en bloque" y centrado
    El teorema de Pitagoras \( a^2 + b^2 = c^2 \) se demostró ser invalido para otros
    exponentes que no sean 2, osea, que no es valido para:
    \[ a^n + b^n = c^n, \qquad \forall n \in \mathbb{R} \]

    Sabemos que:
    \[ ( \ln(x) )' = \frac{1}{x} \]

    Si suponemos que \(x\) es una funcion \(f\) , entonces, por regla de la cadena:
    \[ ( \ln( f(x) ) )' = \frac{1}{ f(x) } \cdot f'(x) = \frac{ f'(x) }{ f(x) } \]

    Una \textbf{\textit{matriz identidad}} es una matriz de dimension \(n \times n\) (es decir,
    cuadrada) que tiene su diagonal principal con 1, y todo el resto 0:
    \[
        I_n =
        \left[\begin{matrix}
            1      & 0      & 0      & \cdots & 0      \\
            0      & 1      & 0      & \cdots & 0      \\
            0      & 0      & 1      & \cdots & 0      \\
            \vdots & \vdots & \vdots & \ddots & \vdots \\
            0      & 0      & 0      & \cdots & 1      \\
        \end{matrix}\right]_{n \times n}
    \]

    Se pueden \textbf{alinear} ecuaciones de la siguiente manera:
    \begin{align*}
        20x + 5 &= \frac{y}{2} \\
        20x &= \frac{y}{2} - 5 \\
        20x &= \frac{y - 10}{2} \\
        x &= \frac{y - 10}{20}
    \end{align*}

\end{document}
