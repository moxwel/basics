\documentclass[12pt]{article}

\usepackage[utf8]{inputenc}
\usepackage[spanish]{babel}
\usepackage[legalpaper]{geometry}

% Para escribir simbolos matematicos, se necesitan estos paquetes
\usepackage{amsmath, amsfonts, amssymb}
% Para poder encuadrar texto de forma bonita, se usa este paquete
\usepackage{fancybox}

\setlength{\parskip}{1em}

\begin{document}

    \LaTeX{} tambien se caracteriza por poder poner expresiones matematicas
    pero de una forma relativamente sencilla.

    % \( ... \) se usa para poner expresiones matematicas "en linea" con el texto
    El teorema de Pitagoras \( a^2 + b^2 = c^2 \) se demostró ser invalido para otros
    exponentes que no sean 2, osea, que no es valido para:
    % \[ ... \] se usa para poner expresiones matematicas "en bloque" y centrado
    \[ a^n + b^n = c^n, \qquad \forall n \in \mathbb{R} \]

    Sabemos que:
    \[ ( \ln(x) )' = \frac{1}{x} \]

    Si suponemos que \(x\) es una funcion \(f\) , entonces, por regla de la cadena:
    \[ ( \ln( f(x) ) )' = \frac{1}{ f(x) } \cdot f'(x) = \frac{ f'(x) }{ f(x) } \]

    Una \textbf{\textit{matriz identidad}} es una matriz de dimension \(n \times n\) (es decir,
    cuadrada) que tiene su diagonal principal con 1, y todo el resto 0:
    \[
        I_n =
        \left[\begin{matrix}
            1      & 0      & 0      & \cdots & 0      \\
            0      & 1      & 0      & \cdots & 0      \\
            0      & 0      & 1      & \cdots & 0      \\
            \vdots & \vdots & \vdots & \ddots & \vdots \\
            0      & 0      & 0      & \cdots & 1
        \end{matrix}\right]_{n \times n}
    \]

    Se puede \textbf{alinear} ecuaciones de la siguiente manera:
    % El simbolo & (ampersand) va a ser el caracter con el que se va a alinear
    % la ecuacion
    \begin{align*}
        20x + 5 &= \frac{y}{2}      \\
            20x &= \frac{y}{2} - 5  \\
            20x &= \frac{y - 10}{2} \\
              x &= \frac{y - 10}{20}
    \end{align*}

    Podemos poner ecuaciones dentro de un cuadrado de esta forma:
    % El comando "shadowbox" nos va a crear un cuadro con sombra, bien bonito.
    % Dentro de shadowbox podemos poner una expresion matematica en linea.
    % Usando el comando "displaystyle", la ecuacion no se "compacta" al texto.
    \[ \shadowbox{\( \displaystyle \sum_{k=1}^{n} k = \frac{n (n - 1)}{2} \)} \]

\end{document}
