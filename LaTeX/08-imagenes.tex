\documentclass[12pt]{article}

\usepackage[utf8]{inputenc}
\usepackage[spanish]{babel}
\usepackage[legalpaper, margin=2cm]{geometry}
\usepackage{amsmath, amsfonts, amssymb}

\setlength{\parskip}{1em}

% Para poder insertar imagenes, se puede usar el siguente paquete
\usepackage{graphicx}
% Aqui se define la carpeta en donde van a estar todas las imagenes
\graphicspath{ {./sources/} }

\begin{document}

    Se pueden insertar imagenes en \LaTeX.

    Por ejemplo, Miguel:

    % Ambito de imagen para poder centrarlo
    \begin{figure}[h]
        \centering % Centrar

        % Con este comando, se puede insertar imagenes, usando el nombre del archivo.
        \includegraphics{miguel.png}

        \caption{Miguel} % Figura 1: Miguel
    \end{figure}

    Saluda a Miguel.

    \rule{5cm}{0.4pt}

    Sigue pagina...

    \newpage


    Miguel puede ser mas chikito.

    \begin{figure}[h]
        \centering

        % Se puede usar el atributo "scale" para cambiar el tamaño de la imagen
        \includegraphics[scale=0.2]{miguel.png}

        \caption{Miguel chikito}
    \end{figure}


    O mas grande.

    \begin{figure}[h]
        \centering

        \includegraphics[scale=2]{miguel.png}

        \caption{Miguelote}
    \end{figure}

    Podemos aplastar a Miguel

    \begin{figure}[h]
        \centering

        % Con los atributos "width" y "height" se puede cambiar el tamaño de forma
        % mas especifica
        \includegraphics[width=400px, height=100px]{miguel.png}

        \caption{Miguel aplastado nooooo}
    \end{figure}

    Presiona \textbf{F} para \textit{\textbf{Dar Respetos}} a Miguel \texttt{u\_u}.

\end{document}
