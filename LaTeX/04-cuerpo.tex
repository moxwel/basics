\documentclass[12pt]{article}

\usepackage[utf8]{inputenc}
\usepackage[spanish]{babel}

\title{Mi primer documento}
\author{Max S.}
\date{\today}

% Este es el cuerpo del documento, aqui se plasma todo el texto que se va a ver
\begin{document}

    % Seccion de portada
    \begin{titlepage}
        \maketitle
    \end{titlepage}

    % Seccion de resumen
    \begin{abstract}
        Este sera una pequeña seccion en donde se mostrara un breve resumen del
        articulo. Como se ve, tiene un diseño diferente.
    \end{abstract}

    Éste es el contenido de nuestro documento \LaTeX. Aqui podemos escribir
    todo nuestro contenido, y las cosas que queremos mostrar. La gracia de \LaTeX
    es que solamente se centra en el contenido, dejando al computador encargarse
    del diseño.

    Tambien podemos hacer saltos de pagina.

    \newpage

    Al dejar un espacio entremedio de las lineas, se crea un nuevo parrafo.
    Tambien podemos romper lineas si \\ utilizamos \\ esto...

    Tambien podemos crear secciones para organizar el contenido de nuestro
    documento:

    \section{Mi seccion}

        Aqui podemos escribir dentro de la seccion 1

        \subsection{Mi subseccion}

        O una seccion dentro de otra seccion.

    \section*{Mi seccion sin numero}

        Podemos quitarle el numero a las secciones si queremos.

        \subsection*{Mi subseccion sin numero}

        Tambien sirve para subsecciones.

\end{document}
