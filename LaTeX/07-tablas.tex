\documentclass[12pt]{article}

\usepackage[utf8]{inputenc}
\usepackage[spanish]{babel}
\usepackage[letterpaper]{geometry}
\usepackage{amsmath, amsfonts, amssymb}

\setlength{\parskip}{1em}

\begin{document}

    Se pueden crear tablas en \LaTeX :

    % Se crea un ambito de tabla para poder centrarlo
    \begin{table}[h]
        \centering
        \begin{tabular}{ c | c | c || c | c }
            % c = centrar los elementos de la celda.
            % | = crear linea para separar cada celda
            % || = crear linea doble para separar
            \(p\)      & \(q\)      & \(\lnot p\) & \(\lnot p \lor q\) & \(p \implies q\) \\
            % Crear linea horizontal
            \hline \hline
            \textbf{V} & \textbf{V} & \textbf{F}  & \textbf{V}         & \textbf{V}       \\
            \textbf{V} & \textbf{F} & \textbf{F}  & \textbf{F}         & \textbf{F}       \\
            \textbf{F} & \textbf{V} & \textbf{V}  & \textbf{V}         & \textbf{V}       \\
            \textbf{F} & \textbf{F} & \textbf{V}  & \textbf{V}         & \textbf{V}       \\
        \end{tabular}

        Por lo tanto se puede decir que \( ( \lnot p \lor q ) \equiv ( p \implies q ) \) .
    \end{table}

\end{document}
