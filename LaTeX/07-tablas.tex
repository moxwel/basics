\documentclass[12pt]{article}

\usepackage[utf8]{inputenc}
\usepackage[spanish]{babel}
\usepackage[letterpaper]{geometry}
\usepackage{amsmath, amsfonts, amssymb}

% Se utiliza este paquete para poner color en tablas
\usepackage{colortbl}

\setlength{\parskip}{1em}

\begin{document}

    Se pueden crear tablas en \LaTeX{}.

    Por ejemplo, para hacer una tabla de verdad logicas:

    % Se crea un ambito de tabla para poder centrarlo
    \begin{table}[h]
        \centering % Centrar

        % c = centrar los elementos de la celda.
        % | = crear linea para separar cada celda
        % || = crear linea doble para separar
        \begin{tabular}{ c | c | c || c | c }
            \(p\)      & \(q\)      & \(\lnot p\) & \(\lnot p \lor q\) & \(p \implies q\) \\
            \hline \hline % Crear doble linea horizontal
            \textbf{V} & \textbf{V} & \textbf{F}  & \textbf{V}         & \textbf{V}       \\
            % "cellcolor" cambia el color de una celda especifica, los argumentos son
            % {color!transparencia(1-100)}
            \rowcolor[rgb]{1,0.6,0.6}
            \textbf{V} & \textbf{F} & \textbf{F}  & \textbf{F}         & \textbf{F}       \\
            \textbf{F} & \textbf{V} & \textbf{V}  & \textbf{V}         & \textbf{V}       \\
            \textbf{F} & \textbf{F} & \textbf{V}  & \textbf{V}         & \textbf{V}
        \end{tabular}
    \end{table}

    Por lo tanto se puede decir que \( ( \lnot p \lor q ) \equiv ( p \implies q ) \) .

\end{document}
