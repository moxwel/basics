\documentclass[12pt]{article}

\usepackage[utf8]{inputenc}
\usepackage[spanish]{babel}

% Con esto se puede cambiar el formato de hoja y margenes del documento
\usepackage[legalpaper, margin=1cm]{geometry}

% Con esto se puede cambiar la separacion entre parrafos
\setlength{\parskip}{1em}

\begin{document}

    % Noindent sirve para quitar la sangria de un parrafo
    \noindent Podemos formatear el texto para dar un \textbf{enfasis} en el
    contenido. Por ejemplo:

    % Comienzo de texto en viñetas
    \begin{itemize}
        % Cada elemento que contenga un "item" tendra una viñeta
        \item \textbf{Texto en negrita}

        \item \textit{Texto en cursiva}

        \item \underline{Texto subrayado}

        \item \texttt{Texto monoespaciado}

        \item Texto en `comillas simples' o ``comillas dobles''
    \end{itemize}

    \noindent Tambien podemos hacer una mezcla de estas:
    \begin{itemize}
        \item \textbf{\textit{Texto en negrita y cursiva}}

        \item \textit{\texttt{Texto en cursiva y monoespaciado}}

        \item \underline{\textit{Texto subrayado y cursiva}}
    \end{itemize}

    \noindent Formateando el texto en \LaTeX, nos servira para:
    % Comienzo de texto enumerado
    \begin{enumerate}
        % Cada elemento que contenga un "item" tendra un numero.
        \item Enfatizar aspectos \textbf{\underline{importantes}} de nuestro texto.

        \item Dar mas \textit{dinamismo} al texto.

        \item No aburrir al lector con texto \texttt{aburrido}.
    \end{enumerate}

\end{document}
